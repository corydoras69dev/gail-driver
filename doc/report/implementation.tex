\chapter{Implementation}
\label{chapter:Implementation}

This chapter describes the main modules written by Python and Julia.
The structure of each neural network is also described.

The following diagrams will be inserted.
\begin{itemize}
\item class diagrams of important classes.
\item relation of data and modules
\end{itemize}




Now writing...


\section{Software structure summary}

The program is roughly divided into a training program consisting of Python and Julia and a validation program consisting only of Julia.

Now writing...


\subsection{Training program summary}

All vehicle behavior is described by the Julia program. The discriminator network is implemented in python's tensorflow.
policy network is a ForwardNets module by Julia, but uses python's tensorflow to configure the network.

Now writing...


\subsection{Validation program summary}

All vehicle behavior is described by the Julia program. 
Policy is also written in Julia.
The validation program does not update Policy. It also doesn't use the discriminator, so it doesn't use Python code.

Now writing...


\section{Features for policy network}


This section details the 51-element input vector used by Policy.

Now writing...

\section{Input for discriminator}


This section details input of discriminator.

Now writing...

\section{Metrics}


This section details the metrics for vehicle behavior that are evaluated by the validation program.

Now writing...

\begin{itemize}
\item Root Weighted Square Error.
\begin{itemize}
\item Position
\item Lane Offset
\item Speed
\end{itemize}
\item Kullback-Leobler Divergence.
\begin{itemize}
\item iTTC
\item Speed
\item Acceleration
\item Turn-Rate
\item Jerk
\end{itemize}
\item Emergent Value.
\begin{itemize}
\item Lane Change Rate
\item Offroad Duration
\item Collision Rate
\item Hard Break Rate
\end{itemize}
\end{itemize}

\section{Summary of environment}

Now writeng...

\begin{itemize}
\item Interstate rode 101 and 80.
\item Data sampling frequency, sampling times.
\end{itemize}


\section{Reward network summary}

Reward network is a discriminator and is realized by tensorflow. The surrogate reward is calculated based on the discrimination result by this network.


Now writing...
\begin{itemize}
\item input, output
\item layers
\item size of nodes
\end{itemize}

\begin{itemize}
\item location in sourcecode.
\item specification
\end{itemize}


\section{Policy network summary}

Now writing...

\begin{itemize}
\item input, output
\item layers
\item size of nodes
\end{itemize}

\subsection{GRU structure}

\begin{itemize}
\item location in sourcecode.
\item specification
\end{itemize}

\subsection{MLP structure}

\begin{itemize}
\item location in sourcecode.
\item specification
\end{itemize}

\section{Summary of external modules for this experiments}

\begin{itemize}
\item AutomotiveDrivingModels.jl
\item NGSIM.jl
\item AutoViz.jl
\item ForwardNets.jl
\item rllab
\item AutoDrivers.jl
\end{itemize}

(Information such as tag id, github address, branch name will be specified in the users guide.)

\section{Changes to the original program}


List of the original program and the changes will be inserted.

